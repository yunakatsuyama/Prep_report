\documentclass[12pt, a4paper]{article}

\usepackage[utf8]{inputenc}
\usepackage[T1]{fontenc}
\usepackage{amsmath}
\usepackage{amssymb}
\usepackage{graphicx}
\usepackage{float}
\usepackage[utf8]{inputenc}
\usepackage{booktabs}
\usepackage[version=4]{mhchem}
\usepackage{siunitx}
\usepackage{subcaption}
\usepackage{hyperref}
\hypersetup{colorlinks=true, linkcolor=blue, citecolor=green, urlcolor=magenta}


\usepackage[style=apa,backend=biber]{biblatex}
\addbibresource{main.bib}


\usepackage[margin=1in]{geometry}

\usepackage{caption}
\usepackage{subcaption}


\author{Yuna Katsuyama }
\date{\today}

\begin{document}

\begin{titlepage}
    \centering
    \begin{minipage}{0.49\textwidth}
        \raggedright
        \includegraphics[width=0.5\textwidth]{logo.png}
    \end{minipage}
    \hfill
    \begin{minipage}{0.49\textwidth}
        \raggedleft
        \includegraphics[width=0.3\textwidth]{iupLogo2.png}
    \end{minipage}

    \vspace{3.5cm}
    
    {\LARGE \textbf{A Study on Super-Resolution Neural Networks for Anthropogenic Emission Data}}\\[2cm]
    
    {\Large \textbf{Preparatory Project
    }}\\[1cm]
    \vspace{0.5cm}
    {\large submitted by}\\[0.5cm]
    
    {\Large \textbf{Yuna Katsuyama}}\\
    \vspace{2cm}
    
    \rule{\textwidth}{0.4mm}\\[0.5cm]
    
    {\large Supervisor}\\[0.3cm]
    {\large Prof. Mihalis Vrekoussis, Dr. Alex Poulidis, Santiago Parraguez Cerda}\\[0.5cm]
    
    \rule{\textwidth}{0.4mm}\\[0.5cm]
    \vspace{0.3cm}
    {\large University of Bremen}\\
    {\large Institute of Environmental Physics}\\[1.5cm]
    { Date of Birth: 23.02.2002} \\
    { Matriculation Number: 6421479} \\
    \vspace{1.0cm}
    {Submission Date: 20.02.26}
    
\end{titlepage}

\thispagestyle{empty}
\clearpage

\begin{abstract}
		Air pollution remains a major global health and environmental challenge, particularly in rapidly urbanizing regions where high-resolution emission data are scarce. With my proposed project I aim to develop a generalized, transferable framework for downscaling coarse global emission inventories to urban-scale resolution (1 km) using conditional Generative Adversarial Networks (cGANs). By learning spatial patterns from auxiliary data (for example road networks, land use, and night-time lights) the model will enable emission downscaling in any region, even without region-specific training. The downscaled emissions will be integrated into a chemistry transport model (e.g., ICON-ART) to simulate urban air quality, identify key pollution sources, and evaluate mitigation strategies for three main target cities, Tokyo (Japan), Hamburg (Germany), and Bangkok (Thailand). The outcome will be an open-source, globally applicable tool that improves the accuracy of air quality modeling, benefiting both high- and low-income countries. My research will address a critical data gap, support evidence-based policy, and advance equitable access to environmental science.
		
\end{abstract}
	
\tableofcontents
\newpage

\section{Introduction}
\subsection{Air pollution}
% People die due to air pollution. (health risks), WHO
Air pollution causes serious health problems. In 2019, air pollution caused about 6.7 million deaths.  Air pollutants affect the cardiovascular, neurological, respiratory and other organ systems and they increases the risk of death from cardiovascular and respiratory disease, and lung cancer.
The major health-damaging air pollutants are particulate matter (PM2.5 and PM10), ozone, nitrogen dioxide, sulfur dioxide and carbon monoxide. (\cite{WHO:2024})

In order to reduce the health impact, WHO published Air Quality Guidelines (AQG) that is the safe levels of pollutant to human health in 1987 and updated them in 2005 and 2021.
(\cite{WHO:2021})

% history of air pollution 
In the past, two big air pollution events which are London smog in 1952 and Los angels smog from 1940s to 1970s occurred. 
Air pollution has been dealt with for centuries and has the air quality has greatly improved. 


% Air quality guidline, and current situation (figure from Fowler )
However, as shown in Figure \ref{fig:air_pollution}, a large proportion of the population still lives in the regions where air quality is worse than the AQG values.
\begin{figure}[H]
    \centering
    \includegraphics[width=0.7\columnwidth]{AQG.jpg}
    \caption{Distributions of the population as a function of annual (2013) average ambient PM2.5 concentration for the world's 10 most populous countries and the rest of the world. Dashed vertical lines indicate World Health Organization Interim Targets (IT) and the Air Quality Guideline (AQG). Reproduced from \cite{Fowler:2020}.}
    \label{fig:air_pollution}
\end{figure}

Furthermore, while air quality has improved markedly in high-income countries over this period, it has generally deteriorated in most low- and middle-income countries, in line with large-scale urbanization and economic development. (\cite{WHO:2021})
This indicates that addressing air pollution is particularly urgent in developing countries.



(different reason of air pollution , out door , in door ) 
% Air pollutant (NOx, PM, ozone, SO2, CO, VOC, NH3)
According to \cite{folberth2015megacities}, the 26 largest cities account for significant shares of global emissions of greenhouse gases, 12\% of carbon dioxide (CO$_2$), 7\% of methane (CH$_4$), and large fractions of major air pollutants such as nitrogen oxides (NO$_X$; 4.6\%), sulphur dioxide (SO$_2$; 5.3\%), Black Carbon (BC; 3.8\%) and Volatile Organic Compounds (VOCs; 4.8\%)
Air pollutants can be grouped into two, primary and secondary pollutants. Primary pollutants are emitted from source and secondary pollutants are chemically produced  from the primary pollutant. 

NO$_X$ is a mixture of NO and NO$_2$. It is a primary pollutant and it mainly emitted from human activities. High-temperature combustion such as transportation, power plants and industry produces ozone from N$_2$ and O$_2$.  
NO$_X$ has a short life time from one day to few days. 
PM is classified two PM$_2.5$ and PM$_10$ by its particle size. 
Example of VOCs are benzen, toluene and formaldehyde. 80\~90 \% of VOCs are emitted from natural sources such as trees, plants and soil. Anthropogenic emission sources are solvent (paint and cleaning product), fuel and gas. 

%     Emission sources of them
%     Chemistry of them (life time, ozone production etc..)
% Mitigation studies using CTM
	Chemistry transport models (CTMs) are the best tool for assessing air-pollution mitigation strategies. They simulate the emission, transport, and chemical transformation of key pollutants (e.g. NO$_X$, SO$_2$, CO, VOCs, and PM) to generate time-dependent, three-dimensional concentration and deposition fields (\cite{seinfeld2016atmospheric}). Pollutant lifetimes depend on meteorology (through wind, temperature, precipitation) and chemical interactions and are further complicated by secondary pollutant formation such as O$_3$, and secondary organic aerosols (SOA). By simulating these processes, CTMs reproduce atmospheric conditions and surface pollutant levels and enable scenario-based analyses (for example for emission reductions) to project air quality changes with potential mitigation strategies (\cite{seinfeld2016atmospheric}).

% Complexty of chemistry (example of ozone)
\subsection{Importance of emission inventories}
%for chemistry transport model
%he accuracy of CTM depends on emisson inventories
%Existing emission inventroes (cams, eagrid, german one, local one..)
\subsection{Emission downscaling}
\subsubsection{Rule-based downscaling}
%Ramacher et al.,2021
\subsubsection{Machine learning-based downscaling}
%(ふわっとした説明のみ)

\subsection{Purposes}
%My purpose of my project (entire of my master)
In my master project, I aim to develop a downscale method of emission inventories by using machine learning. Implementation of auxiliary data (for example road networks, population distribution, land use and night-time lights) will be expected to improve the downscale accuracy. And city-scale analysis of downscaled emission inventories will enable to investigate the cause of air pollution.  
% Explain on preparatory project
In my preparatory project, I aim to make initial result of emission downscaling using machine learning. The project consists of 3 main tasks, air quality data manipulation, neural network development and diagnosis with the sample dataset, and initial test with actual air quality data manipulation.  


\section{Methods}
\subsection{Machine learning models}
Machine learning is a method to make computers learn from data without explicit programming. Basement of machine learning is linear regression. Neural network is a machine learning architecture. 
Convolutional neural networks (CNNs) are a supervised machine learning method, which were originally developed for images (2D data). They consist of a neural network with additional layers called convolution layers and pooling layers. A convolution layer extracts patterns from an image by applying a 2D filter (kernel) whose values are learnable parameters. The pooling layer then reduces the spatial dimensions of the resulting feature maps, which helps retain the most important information while improving robustness against slight translations, rotations, and small distortions in the input image.
Super-resolution CNNs (SRCNNs) represents the mapping function that takes the low-resolution image as the input and outputs the high-resolution one. It consists of three operations: patch extraction and representation, non-linear mapping and reconstruction. These three operators are modeled by the same form of a convolutional layer. (\cite{dong2014srcnn}).

\subsection{Evaluation metrix}
Loss function is an index of the model performance and it is calculated by the difference between the model result and target values. Loss function is also used to train the models by back error propagation. Two major ways are by mean square error (MSE) and binary cross entropy (BCE). Suitable adaptation of loss function is important for the effective training. MSE is for prediction and BCE is for classification task.

\subsection{Experiment1: Hand written digits classification}
\subsection{Experiment2: Image super resolution}
\subsection{Experiment3: Emission downscaling}


%使ったモデルの説明
% Experimental design 
% HOw many time of epoch, loss , SDG, or 
% 
% sample image of handwritten, sr images
% の説明
% Cams, EAgrid
Global emission inventories typically have coarse spatial resolutions, over 0.1$^\circ$ ($\sim10~km$). The Copernicus Atmosphere Monitoring Service global anthropogenic dataset (CAMS-GLOB-ANT; \cite{Kuenen2022}) is the current state of the art, covering the globe with a 0.1$^\circ$ resolution.

% Interpolation のdata を見せる
% 
\subsection{Regrid method}
area weighted average

\section{Preliminary Results}
% = 図を見たらわかることを述べる (事実)
\subsection{CNN with sample dataset}
% one example
% -> diagnostics
\subsection{SRCNN with sample dataset}
\subsection{SRCNN with cams and eagrid}
\section{Discussion}
% discussion はresultからわかること
% Discussion とintroduction で述べたquestiomn とlink していないいけない。
% discussionでは、複数のresult から考えたことを書く
\section{Conclusion}
\section{Future Steps}





\printbibliography

\end{document}