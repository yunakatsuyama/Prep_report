\documentclass[12pt, a4paper]{article}

\usepackage[utf8]{inputenc}
\usepackage[T1]{fontenc}
\usepackage{amsmath}
\usepackage{amssymb}
\usepackage{graphicx}
\usepackage{float}
\usepackage[utf8]{inputenc}
\usepackage{booktabs}
\usepackage[version=4]{mhchem}
\usepackage{siunitx}
\usepackage{subcaption}
\usepackage{hyperref}
\hypersetup{colorlinks=true, linkcolor=blue, citecolor=green, urlcolor=magenta}


\usepackage[style=apa,backend=biber]{biblatex}
\addbibresource{main.bib}


\usepackage[margin=1in]{geometry}

\usepackage{caption}
\usepackage{subcaption}


\author{Yuna Katsuyama }
\date{\today}

\begin{document}

\begin{titlepage}
    \centering
    \begin{minipage}{0.49\textwidth}
        \raggedright
        \includegraphics[width=0.5\textwidth]{logo.png}
    \end{minipage}
    \hfill
    \begin{minipage}{0.49\textwidth}
        \raggedleft
        \includegraphics[width=0.3\textwidth]{iupLogo2.png}
    \end{minipage}

    \vspace{3.5cm}
    
    {\LARGE \textbf{A Study on Super-Resolution Neural Networks for Anthropogenic Emission Data}}\\[2cm]
    
    {\Large \textbf{Preparatory Project
    }}\\[1cm]
    \vspace{0.5cm}
    {\large submitted by}\\[0.5cm]
    
    {\Large \textbf{Yuna Katsuyama}}\\
    \vspace{2cm}
    
    \rule{\textwidth}{0.4mm}\\[0.5cm]
    
    {\large Supervisor}\\[0.3cm]
    {\large Prof. Mihalis Vrekoussis, Dr. Alex Poulidis, Santiago Parraguez Cerda}\\[0.5cm]
    
    \rule{\textwidth}{0.4mm}\\[0.5cm]
    \vspace{0.3cm}
    {\large University of Bremen}\\
    {\large Institute of Environmental Physics}\\[1.5cm]
    
    {\large Date: 20.02.26}
    
\end{titlepage}

\thispagestyle{empty}
\clearpage

\begin{abstract}
		Air pollution remains a major global health and environmental challenge, particularly in rapidly urbanizing regions where high-resolution emission data are scarce. With my proposed project I aim to develop a generalized, transferable framework for downscaling coarse global emission inventories to urban-scale resolution (1 km) using conditional Generative Adversarial Networks (cGANs). By learning spatial patterns from auxiliary data (for example road networks, land use, and night-time lights) the model will enable emission downscaling in any region, even without region-specific training. The downscaled emissions will be integrated into a chemistry transport model (e.g., ICON-ART) to simulate urban air quality, identify key pollution sources, and evaluate mitigation strategies for three main target cities, Tokyo (Japan), Hamburg (Germany), and Bangkok (Thailand). The outcome will be an open-source, globally applicable tool that improves the accuracy of air quality modeling, benefiting both high- and low-income countries. My research will address a critical data gap, support evidence-based policy, and advance equitable access to environmental science.
		
\end{abstract}
	
\tableofcontents
\newpage

\section{Introduction}

I am tired (\cite{WHO:2024})
I need coffee (\cite{Breiman1996})

\section{Methods}


\section{Preliminary Results}

\section{Discussion}

\section{Conclusion}





\printbibliography

\end{document}